\section{Conclusions}

Using Alexa as a digital assistant helps users easily find products in a mini-market, saving both time and effort. By allowing shoppers to quickly ask about product availability without having to search manually, Alexa improves the overall shopping experience and makes it more convenient.

Additionally, using Amazon Web Services (AWS) technologies to implement computer vision with Alexa is both time-efficient and cost-effective. AWS provides a wide range of tools and services that make it easier to develop and deploy these kinds of applications. The comprehensive documentation and scalability of AWS also help reduce the time and resources needed to get the system up and running.

One of the key factors in building a successful model is creating a high-quality dataset. The variety, quality, and quantity of data used for training the model play an important role in its performance. A well-balanced dataset ensures that the model can recognize products accurately and work reliably in real-world situations. 

For future work, we recommend focusing on improving the dataset to enhance model accuracy and performance. Additionally, finding more efficient ways to organize and dispose of products could lead to better results. Another important consideration is the cost of the AWS services used in this project. While the costs of Alexa, DynamoDB, and Lambda are nearly zero, the main expense comes from the computer vision solution, which needs to be optimized for lower operational costs. By refining these elements, future iterations of the system can be more efficient and cost-effective while delivering better performance. This solution also helps preventing inconsistencies between the database and real world, sometimes caused by Human error. 